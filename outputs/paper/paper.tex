% Options for packages loaded elsewhere
\PassOptionsToPackage{unicode}{hyperref}
\PassOptionsToPackage{hyphens}{url}
\PassOptionsToPackage{dvipsnames,svgnames,x11names}{xcolor}
%
\documentclass[
  letterpaper,
  DIV=11,
  numbers=noendperiod]{scrartcl}

\usepackage{amsmath,amssymb}
\usepackage{iftex}
\ifPDFTeX
  \usepackage[T1]{fontenc}
  \usepackage[utf8]{inputenc}
  \usepackage{textcomp} % provide euro and other symbols
\else % if luatex or xetex
  \usepackage{unicode-math}
  \defaultfontfeatures{Scale=MatchLowercase}
  \defaultfontfeatures[\rmfamily]{Ligatures=TeX,Scale=1}
\fi
\usepackage{lmodern}
\ifPDFTeX\else  
    % xetex/luatex font selection
\fi
% Use upquote if available, for straight quotes in verbatim environments
\IfFileExists{upquote.sty}{\usepackage{upquote}}{}
\IfFileExists{microtype.sty}{% use microtype if available
  \usepackage[]{microtype}
  \UseMicrotypeSet[protrusion]{basicmath} % disable protrusion for tt fonts
}{}
\makeatletter
\@ifundefined{KOMAClassName}{% if non-KOMA class
  \IfFileExists{parskip.sty}{%
    \usepackage{parskip}
  }{% else
    \setlength{\parindent}{0pt}
    \setlength{\parskip}{6pt plus 2pt minus 1pt}}
}{% if KOMA class
  \KOMAoptions{parskip=half}}
\makeatother
\usepackage{xcolor}
\setlength{\emergencystretch}{3em} % prevent overfull lines
\setcounter{secnumdepth}{5}
% Make \paragraph and \subparagraph free-standing
\ifx\paragraph\undefined\else
  \let\oldparagraph\paragraph
  \renewcommand{\paragraph}[1]{\oldparagraph{#1}\mbox{}}
\fi
\ifx\subparagraph\undefined\else
  \let\oldsubparagraph\subparagraph
  \renewcommand{\subparagraph}[1]{\oldsubparagraph{#1}\mbox{}}
\fi


\providecommand{\tightlist}{%
  \setlength{\itemsep}{0pt}\setlength{\parskip}{0pt}}\usepackage{longtable,booktabs,array}
\usepackage{calc} % for calculating minipage widths
% Correct order of tables after \paragraph or \subparagraph
\usepackage{etoolbox}
\makeatletter
\patchcmd\longtable{\par}{\if@noskipsec\mbox{}\fi\par}{}{}
\makeatother
% Allow footnotes in longtable head/foot
\IfFileExists{footnotehyper.sty}{\usepackage{footnotehyper}}{\usepackage{footnote}}
\makesavenoteenv{longtable}
\usepackage{graphicx}
\makeatletter
\def\maxwidth{\ifdim\Gin@nat@width>\linewidth\linewidth\else\Gin@nat@width\fi}
\def\maxheight{\ifdim\Gin@nat@height>\textheight\textheight\else\Gin@nat@height\fi}
\makeatother
% Scale images if necessary, so that they will not overflow the page
% margins by default, and it is still possible to overwrite the defaults
% using explicit options in \includegraphics[width, height, ...]{}
\setkeys{Gin}{width=\maxwidth,height=\maxheight,keepaspectratio}
% Set default figure placement to htbp
\makeatletter
\def\fps@figure{htbp}
\makeatother
% definitions for citeproc citations
\NewDocumentCommand\citeproctext{}{}
\NewDocumentCommand\citeproc{mm}{%
  \begingroup\def\citeproctext{#2}\cite{#1}\endgroup}
\makeatletter
 % allow citations to break across lines
 \let\@cite@ofmt\@firstofone
 % avoid brackets around text for \cite:
 \def\@biblabel#1{}
 \def\@cite#1#2{{#1\if@tempswa , #2\fi}}
\makeatother
\newlength{\cslhangindent}
\setlength{\cslhangindent}{1.5em}
\newlength{\csllabelwidth}
\setlength{\csllabelwidth}{3em}
\newenvironment{CSLReferences}[2] % #1 hanging-indent, #2 entry-spacing
 {\begin{list}{}{%
  \setlength{\itemindent}{0pt}
  \setlength{\leftmargin}{0pt}
  \setlength{\parsep}{0pt}
  % turn on hanging indent if param 1 is 1
  \ifodd #1
   \setlength{\leftmargin}{\cslhangindent}
   \setlength{\itemindent}{-1\cslhangindent}
  \fi
  % set entry spacing
  \setlength{\itemsep}{#2\baselineskip}}}
 {\end{list}}
\usepackage{calc}
\newcommand{\CSLBlock}[1]{\hfill\break\parbox[t]{\linewidth}{\strut\ignorespaces#1\strut}}
\newcommand{\CSLLeftMargin}[1]{\parbox[t]{\csllabelwidth}{\strut#1\strut}}
\newcommand{\CSLRightInline}[1]{\parbox[t]{\linewidth - \csllabelwidth}{\strut#1\strut}}
\newcommand{\CSLIndent}[1]{\hspace{\cslhangindent}#1}

\KOMAoption{captions}{tableheading}
\makeatletter
\@ifpackageloaded{caption}{}{\usepackage{caption}}
\AtBeginDocument{%
\ifdefined\contentsname
  \renewcommand*\contentsname{Table of contents}
\else
  \newcommand\contentsname{Table of contents}
\fi
\ifdefined\listfigurename
  \renewcommand*\listfigurename{List of Figures}
\else
  \newcommand\listfigurename{List of Figures}
\fi
\ifdefined\listtablename
  \renewcommand*\listtablename{List of Tables}
\else
  \newcommand\listtablename{List of Tables}
\fi
\ifdefined\figurename
  \renewcommand*\figurename{Figure}
\else
  \newcommand\figurename{Figure}
\fi
\ifdefined\tablename
  \renewcommand*\tablename{Table}
\else
  \newcommand\tablename{Table}
\fi
}
\@ifpackageloaded{float}{}{\usepackage{float}}
\floatstyle{ruled}
\@ifundefined{c@chapter}{\newfloat{codelisting}{h}{lop}}{\newfloat{codelisting}{h}{lop}[chapter]}
\floatname{codelisting}{Listing}
\newcommand*\listoflistings{\listof{codelisting}{List of Listings}}
\makeatother
\makeatletter
\makeatother
\makeatletter
\@ifpackageloaded{caption}{}{\usepackage{caption}}
\@ifpackageloaded{subcaption}{}{\usepackage{subcaption}}
\makeatother
\ifLuaTeX
  \usepackage{selnolig}  % disable illegal ligatures
\fi
\usepackage{bookmark}

\IfFileExists{xurl.sty}{\usepackage{xurl}}{} % add URL line breaks if available
\urlstyle{same} % disable monospaced font for URLs
\hypersetup{
  pdftitle={How Household Income Impacts Political Views, Focusing on Mr.~Donald Trump, Yielded Inconclusive Results},
  pdfauthor={Gavin Crooks; Samarth Rajani},
  colorlinks=true,
  linkcolor={blue},
  filecolor={Maroon},
  citecolor={Blue},
  urlcolor={Blue},
  pdfcreator={LaTeX via pandoc}}

\title{How Household Income Impacts Political Views, Focusing on
Mr.~Donald Trump, Yielded Inconclusive Results\thanks{Code and data are
available at:
LINK.https://github.com/Crooksyyy/The-Effects-of-Social-Media}}
\author{Gavin Crooks \and Samarth Rajani}
\date{February 15, 2024}

\begin{document}
\maketitle
\begin{abstract}
This study explores the relationship between household income,
ethnicity, and political preferences using data from `The Welfare
Effects of Social Media.' Despite inconclusive findings challenging the
presumed impact of income on political views, our innovative metric,
measuring respondents' propensity to follow Donald Trump, did not
provide conclusive results. This introduces new avenues in political
science and statistics, encouraging further exploration into
understanding the dynamics of these relationships. The study contributes
to the ongoing discourse on the complex relationships between income,
ethnicity, and political inclinations.
\end{abstract}

\renewcommand*\contentsname{Table of contents}
{
\hypersetup{linkcolor=}
\setcounter{tocdepth}{3}
\tableofcontents
}
\section{Introduction}\label{sec-intro}

Household income is defined as the gross income earned by all members in
a household above 15 years of age (SCOTT 2024). Over the years, it has
been debated whether household incomes at all affect one's affiliation
towards a political school of thought. It is a reasonable hypothesis to
assume a sort of relationship between income and voting either Democrat
or Republican, as both parties have different economic outlooks thereby
affecting incomes differently. Maybe higher income inequality polarizes
political leaning further. Therefore, it is in our best interests to
study whether the poor vote to improve their quality of life.

In `Income Inequality and Partisan Voting in the United States', Andrew
Gelman, Lane Kenworthy and Yu-Sung Su (Gelman, Kenworthy, and Su 2010)
make a case for higher earning Americans voting Republican, whereas Jeff
Madrick (Madrick 2020) argues how working-class Americans voted against
their interests in voting Republican. Conflicting theories have emerged,
and we intend on tackling this issue at hand of whether different income
brackets tend to vote differently. By measuring the difference in
propensity to follow Mr.~Donald Trump in relation to household income.

The remainder of this paper is structured as follows.
Section~\ref{sec-data} will introduce the data set and the variables it
contains. Section~\ref{sec-results} will display the findings of our
data in relation to our paper. The \textbf{?@sec-disc} will discuss why
the findings matter and the weaknesses of our analysis.

Our data has been obtained from `The Welfare Effects of Social Media'
(Allcott et al. 2020) . Our code is supported by the following packages
(R Core Team 2022) (Wickham et al. 2019) (Müller 2020) (Xie 2023)

\section{Data}\label{sec-data}

\subsection{Data Introduction}\label{sec-dataintro}

The data for this paper was collected from the replication package of
the paper `The Welfare Effects of Social Media'(Allcott et al. 2020) The
authors from that data had collected this data themselves using an
online survey platform called Qulatrics, inquiring about personal
information such as personal names, IP addresses, extent of by which the
subjects follow politics, etc. Notably the dataset would contain a lot
of confidential information, that if released in the replication package
would cause ethical problems. As a result, the authors included the
de-identified versions of their data collected, which was used in our
analysis too.

\subsection{Income Data}\label{sec-income_data}

Our key variables of interest include household income. The survey
participants were offered options in bins starting at 0 to 9,999 US
dollar range, with every succeeding bin also being 9,999 USD wide. The
bins went up to a ceiling of 50,000 USD per annum, then every next bin
was 25,000 USD wide until a ceiling of 150,000 USD. In our analysis we
combine the bins above 9,999 into -- 20,000 to 49,999, 50,000 to 99,999
and the rest being 100,000 and up. The option of `Prefer not to answer'
was also available, and the entries with that response were dropped. The
distribution of the data can be seen in Figure~\ref{fig-figure1}.
Figure~\ref{fig-figure1} closely resembles the expected distribution of
USA household income. As expected in any income distribution the
majority of responses fall within the average income ranges of the USA,
between 20,000USD and 100,000USD (2022). These factors indicate that the
data set has an accurate representation of household income.

\subsection{Politics Data}\label{sec-pol_data}

Another variable of use is the extent to which the subjects follow
Mr.~Donald Trump, leader of the Republican Party. The possible responses
were `Not at all closely', `Somewhat closely,' `Rather closely' and
`Very Closely,' and the respondents could select one of these options
which will become our measure of measuring subscription to Republican
ideas. This data is represented using a pie chart in
Figure~\ref{fig-figure2}. This variable faces many problems as this
categorical scale is not consistent across respondents. To be specific
we mean someone who identifies as someone who does follows Not at all
closely can be following Mr.~Donald Trump more than someone who
identifies as Somewhat closely. This is a measurement issue within to
the questions asked in the survey and all self-identifying variables in
general. (\textbf{figure2?}) shows the quantity of respondents in each
group. The most common response is that they follow somewhat closely,
and the other responses are relatively even.

\subsection{Ethnicity Data}\label{sec-race_data}

The second question of the data that we have included in our analysis is
the ethnicity of the respondents. This again is a categorical variable,
that the individual self identifies their own ethnicity. The response
options included Asian or Pacific Islander, White / Caucasian, Hispanic,
Black or African American and other. Table~\ref{tbl-race_dist} shows the
percentage of respondents in each with the overwhelming majority of
responses being Caucasian at nearly 70\%. This is actually less than the
most recent estimates by the United States government which estimate
over 75\% of the population is Caucasian (2022). The data also has an
over representation of Asian and Native Americans. This results in an
under representation of Hispanic and African American populations.

\begin{longtable}[]{@{}lr@{}}

\caption{\label{tbl-race\_dist}Percentage of each Ethnicity from
Responses in the facebook ad}

\tabularnewline

\toprule\noalign{}
Ethnicity & Percentage of Responses \\
\midrule\noalign{}
\endhead
\bottomrule\noalign{}
\endlastfoot
American Indian or Alaskan Native & 0.7554138 \\
Asian or Pacific Islander & 13.5806614 \\
Black or African American & 6.0936713 \\
Hispanic & 8.0577472 \\
Other (please specify) & 2.5851939 \\
White / Caucasian & 68.9273124 \\

\end{longtable}

\section{Results}\label{sec-results}

This paper's goal was to identify if lower-income households voted
against their interest. To understand this relationship, we used the
variable for how closely a respondent follows Mr.~Donald Trump as our
measurement of subscription to Republican ideas. Using this measurement,
we organized the proportion of individuals by income class to how
closely they follow Mr.~Donald Trump in Figure~\ref{fig-figure3}. This
graph shows that a proportional amount of each income class follows
Mr.~Donald Trump at similar levels across all income levels.
Specifically, we mean the percentage of individuals following Mr.~Donald
Trump at different levels is the same no matter the income class. This
means we cannot conclude that income class impacts how closely
individuals follow Mr.~Donald Trump, and therefore, cannot conclude that
different income levels subscribe to republican ideas more than others.

\begin{figure}

\centering{

\includegraphics{paper_files/figure-pdf/fig-figure3-1.pdf}

}

\caption{\label{fig-figure3}Number of Respondents who follow Donald
Trump at different levels by Household Income}

\end{figure}%

To further our analysis, we computed the same graph however organized by
race, not income Figure~\ref{fig-figure4}. This resulted in a similar
result as race is proportional between all levels of following
Mr.~Donald Trump. Therefore, consistent across races to follow
republican rhetoric. Again, this means the percentage of people who
follow Mr.~Donald Trump at different levels is the same for each race.
This is more difficult to conclude for minorities as their
representation within the data set is so small as stated in
Section~\ref{sec-race_data}.

\begin{figure}

\centering{

\includegraphics{paper_files/figure-pdf/fig-figure4-1.pdf}

}

\caption{\label{fig-figure4}Number of Respondents who follow Donald
Trump at different levels by Race}

\end{figure}%

Overall, the results of this analysis were inconclusive to measure how
income impacts individuals' propensity to follow republican rhetoric.
There are many reasons this could be true and as stated in the
(\textbf{intro?}) there are multiple schools of thought previously
studied on the topic.

\section{Discussion \{sec-disc\}}\label{discussion-sec-disc}

\subsection{Inconclusive -- not a futile
exercise}\label{inconclusive-not-a-futile-exercise}

Our results show no evident relationship between income and Republic
following. This result is important in of itself -- the insignificant
relationship suggests income does not influence an individual's choice
of presidential candidate. There can be many possible reasons. The first
reason is that other factors apart from income influence this choice to
a much greater extent. For example, a family of 5 kids may be more
influenced by a candidate promising better institutional infrastructure
like more schools and subsidized education. As mentioned in `Factors
Influencing Voting Decision: A Comprehensive Literature Review'
(Kulachai, Lerdtomornsakul, and Homyamyen 2023), the main headings
influencing decisions include `demographic factors, psychological
factors, sociocultural factors and economic factors.' Income is only one
of the many economic factors, and so may have a very small influence on
voting preference.

\subsection{Weaknesses}\label{weaknesses}

The analysis in the paper is not perfect, with some changes that can
make it much more robust. One of the limitations is the race question --
the limitation of only the choice in the survey means individuals who
part of any of those races are not included are unable to provide a data
point. If some of these groups significantly change their voting pattern
based on income, their contribution is lost. This is a classic case of
sampling bias. Secondly, any values that were missing from the dataset
were dropped for this paper. Just as stated previously, the contribution
of those data points is lost, making our conclusions less precise.
Lastly, the survey collected data via an online platform, thereby
leaving out opinions of individuals either not in possession of
technology or not very adept at operating it such as the elderly. This
is another sampling bias result, as it leaves out another key societal
group for the analysis. Thirdly, the income variable only being measured
as a categorical variable limited our analysis. Income is usually
measured as a continous variable which would have allowed analysis such
as regression to be used.

\section*{Appendix}\label{appendix}
\addcontentsline{toc}{section}{Appendix}

\subsection{Figures}\label{figures}

\begin{figure}

\centering{

\includegraphics{paper_files/figure-pdf/fig-figure1-1.pdf}

}

\caption{\label{fig-figure1}Distribution of Income from Responses in the
facebook ad}

\end{figure}%

\begin{figure}

\centering{

\includegraphics{paper_files/figure-pdf/fig-figure2-1.pdf}

}

\caption{\label{fig-figure2}How Closely People Follow Mr.Donald Trump
from Responses in the facebook ad}

\end{figure}%

\subsection{Data Cleaning}\label{data-cleaning}

The data cleaning for this analysis was very straightforward. Once the
data was obtained using the links mentioned in the paper or the github
repo, the variables reqired for the analysis we compiled and any
response with missing data was removed.

\newpage

\section*{References}\label{references}
\addcontentsline{toc}{section}{References}

\phantomsection\label{refs}
\begin{CSLReferences}{1}{0}
\bibitem[\citeproctext]{ref-USA}
2022. \emph{U.S. Census Bureau Quickfacts: United States}.
\url{https://www.census.gov/quickfacts/fact/table/US/PST045222}.

\bibitem[\citeproctext]{ref-paper}
Allcott, Hunt, Luca Braghieri, Sarah Eichmeyer, and Matthew Gentzkow.
2020. {``The Welfare Effects of Social Media.''} \emph{American Economic
Review}. \url{https://doi.org/10.1257/aer.20190658}.

\bibitem[\citeproctext]{ref-lane}
Gelman, Andrew, Lane Kenworthy, and Yu-Sung Su. 2010. {``Income
Inequality and Partisan Voting in the United States.''} \emph{Social
Science Quarterly}. {[}University of Texas Press, Wiley{]}.
\url{http://www.jstor.org/stable/42956457}.

\bibitem[\citeproctext]{ref-mdpi}
Kulachai, Waiphot, Unisa Lerdtomornsakul, and Patipol Homyamyen. 2023.
{``Factors Influencing Voting Decision: A Comprehensive Literature
Review.''} \emph{Social Sciences}.
\url{https://doi.org/10.3390/socsci12090469}.

\bibitem[\citeproctext]{ref-Madrick_2020}
Madrick, Jeff. 2020. {``Why the Working Class Votes Against Its Economic
Interests.''} \emph{The New York Times}, July.
\url{https://www.nytimes.com/2020/07/31/books/review/the-system-robert-reich-break-em-up-zephyr-teachout.html}.

\bibitem[\citeproctext]{ref-her}
Müller, Kirill. 2020. {``Here: A Simpler Way to Find Your Files.''}
\url{https://CRAN.R-project.org/package=here}.

\bibitem[\citeproctext]{ref-citeR}
R Core Team. 2022. \emph{R: A Language and Environment for Statistical
Computing}. Vienna, Austria: R Foundation for Statistical Computing.
\url{https://www.R-project.org/}.

\bibitem[\citeproctext]{ref-hhld}
SCOTT, MICHELLE P. 2024. {``Household Income.''} investopedia.
\url{https://www.investopedia.com/terms/h/household_income.asp}.

\bibitem[\citeproctext]{ref-tidy}
Wickham, Hadley, Mara Averick, Jennifer Bryan, Winston Chang, Lucy
D'Agostino McGowan, Romain François, Garrett Grolemund, et al. 2019.
{``Welcome to the {tidyverse}.''} \emph{Journal of Open Source
Software}. \url{https://doi.org/10.21105/joss.01686}.

\bibitem[\citeproctext]{ref-knitr}
Xie, Yihui. 2023. {``Knitr: A General-Purpose Package for Dynamic Report
Generation in r.''} \url{https://yihui.org/knitr/}.

\end{CSLReferences}



\end{document}
